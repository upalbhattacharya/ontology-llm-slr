\documentclass[a4paper,colorinlistoftodos]{article}
%%%%%%%%%%%%%%%%%%%%%%%%%%%%%%%%%%%%%%%%%%%%%%%%%%%%%%%%%%%%%%%%%%%%%%
% Package Loading
%%%%%%%%%%%%%%%%%%%%%%%%%%%%%%%%%%%%%%%%%%%%%%%%%%%%%%%%%%%%%%%%%%%%%%

%%%%%%%%%% Basic %%%%%%%%%%
\usepackage[utf8]{inputenc}
\usepackage[T1]{fontenc}
\usepackage[displaymath,mathlines,running]{lineno}

%%%%%%%%%% Referencing %%%%%%%%%%
\usepackage[numbers,sort&compress]{natbib}

%%%%%%%%%% Images %%%%%%%%%%
\usepackage{graphicx}
\usepackage{subcaption}      % For subfigures

%%%%%%%%%% Formatting %%%%%%%%%%
\usepackage{longtable}
\usepackage[hidelinks]{hyperref}
\hypersetup{
  colorlinks = true,
  linkcolor = cat-latte-red}
\usepackage{listings}

%%%%%%%%%% Math %%%%%%%%%%
\usepackage{amsmath}
\usepackage{amssymb}
\usepackage{amsthm}

%%%%%%%%%% Presentation %%%%%%%%%%
\usepackage[dvipsnames]{xcolor}
\usepackage[normalem]{ulem}
\usepackage{tcolorbox}
\usepackage{soul}


%%%%%%%%%% Misc %%%%%%%%%%
\usepackage{todonotes}

\newcounter{todocounter}
\newcommand{\todonum}[2][]
{\stepcounter{todocounter}\todo[#1]{\thetodocounter: #2}}

\usepackage[commandnameprefix=always]{changes}
\definechangesauthor[name='Upal Bhattacharya', color=cat-latte-purple]{UB}

%%%%%%%%%%%%%%%%%%%%%%%%%%%%%%%%%%%%%%%%%%%%%%%%%%%%%%%%%%%%%%%%%%%%%%
% Customization
%%%%%%%%%%%%%%%%%%%%%%%%%%%%%%%%%%%%%%%%%%%%%%%%%%%%%%%%%%%%%%%%%%%%%%
\tcbuselibrary{theorems}

%%%%%%%%%% Colors (Catppuccin Latte) %%%%%%%%%%
\definecolor{cat-latte-green}{HTML}{40A02B}
\definecolor{cat-latte-orange}{HTML}{FE640B}
\definecolor{cat-latte-yellow}{HTML}{DF8E1D}
\definecolor{cat-latte-red}{HTML}{D20F39}
\definecolor{cat-latte-blue}{HTML}{04A5E5}
\definecolor{cat-latte-gray}{HTML}{7c7f93}
\definecolor{cat-latte-purple}{HTML}{8839EF}
\definecolor{cat-latte-lavendar}{HTML}{7287fd}

\colorlet{soulred}{cat-latte-red!50}
\colorlet{soulgreen}{cat-latte-green!50}
\colorlet{soulyellow}{cat-latte-yellow!50}
\colorlet{soulorange}{cat-latte-orange!50}
\colorlet{soulblue}{cat-latte-blue!50}
\colorlet{soulgray}{cat-latte-gray!50}

%%%%%%%%%% Callouts %%%%%%%%%%
\newtcbtheorem[auto counter]{caution}{Caution}%
{colback=cat-latte-orange!15,colframe=cat-latte-orange,fonttitle=\bfseries}{caution}
\newtcbtheorem[auto counter]{positive}{Positive}%
{colback=cat-latte-green!15,colframe=cat-latte-green,fonttitle=\bfseries}{positive}
\newtcbtheorem[auto counter]{query}{Query}%
{colback=cat-latte-yellow!15,colframe=cat-latte-yellow,fonttitle=\bfseries}{query}
\newtcbtheorem[auto counter]{answer}{Answer}%
{colback=cat-latte-blue!15,colframe=cat-latte-blue,fonttitle=\bfseries}{answer}
\newtcbtheorem[auto counter]{negative}{Negative}%
{colback=cat-latte-red!15,colframe=cat-latte-red,fonttitle=\bfseries}{negative}
\newtcbtheorem[auto counter]{gap}{Gap}%
{colback=cat-latte-gray!15,colframe=cat-latte-gray,fonttitle=\bfseries}{gap}
\newtcbtheorem[no counter]{upal}{Upal}%
{colback=cat-latte-purple!15,colframe=cat-latte-purple,fonttitle=\bfseries}{upal}
\newtcbtheorem[no counter]{project}{Project}%
{colback=cat-latte-lavendar!15,colframe=cat-latte-lavendar,fonttitle=\bfseries}{project}

%%%%%%%%%% Highlights %%%%%%%%%%
\newcommand{\hlred}[1]{\sethlcolor{soulred}\hl{#1}}
\newcommand{\hlgreen}[1]{\sethlcolor{soulgreen}\hl{#1}}
\newcommand{\hlyellow}[1]{\sethlcolor{soulyellow}\hl{#1}}
\newcommand{\hlblue}[1]{\sethlcolor{soulblue}\hl{#1}}
\newcommand{\hlorange}[1]{\sethlcolor{soulorange}\hl{#1}}
\newcommand{\hlgray}[1]{\sethlcolor{soulgray}\hl{#1}}

%%%%%%%%%% Underline %%%%%%%%%%
\setul{}{1pt} % Width
\newcommand{\ulred}[1]{\setulcolor{cat-latte-red}\ul{#1}}
\newcommand{\ulgreen}[1]{\setulcolor{cat-latte-green}\ul{#1}}
\newcommand{\ulyellow}[1]{\setulcolor{cat-latte-yellow}\ul{#1}}
\newcommand{\ulblue}[1]{\setulcolor{cat-latte-blue}\ul{#1}}
\newcommand{\ulorange}[1]{\setulcolor{cat-latte-orange}\ul{#1}}
\newcommand{\ulgray}[1]{\setulcolor{cat-latte-gray}\ul{#1}}

%%%%%%%%%%%%%%%%%%%%%%%%%%%%%%%%%%%%%%%%%%%%%%%%%%%%%%%%%%%%%%%%%%%%%%
% Project-Specific
%%%%%%%%%%%%%%%%%%%%%%%%%%%%%%%%%%%%%%%%%%%%%%%%%%%%%%%%%%%%%%%%%%%%%%

%%%%%%%%%% Imports and Commands %%%%%%%%%%

%%%%%%%%%% Customization %%%%%%%%%%
 % For common headers, utilities and styling
\usepackage{environ}

\newif\ifhide
\hidetrue % toggle if necessary

% Callouts
\ifhide
  \NewEnviron{hide}{}
  \let\caution\hide
  \let\endcaution\endhide
  
  \let\positive\hide
  \let\endpositive\endhide
  
  \let\negative\hide
  \let\endnegative\endhide
  
  \let\gap\hide
  \let\endgap\endhide
  
  \let\upal\hide
  \let\endupal\endhide
  
  \let\query\hide
  \let\endquery\endhide
  
  \let\answer\hide
  \let\endanswer\endhide
  
  \let\project\hide
  \let\endproject\endhide
\fi

% Highlights
\renewcommand\hlred[1]{#1}
\renewcommand\hlgreen[1]{#1}
\renewcommand\hlblue[1]{#1}
\renewcommand\hlyellow[1]{#1}
\renewcommand\hlorange[1]{#1}
\renewcommand\hlgray[1]{#1}

% Underline
\renewcommand\ulred[1]{#1}
\renewcommand\ulgreen[1]{#1}
\renewcommand\ulblue[1]{#1}
\renewcommand\ulyellow[1]{#1}
\renewcommand\ulorange[1]{#1}
\renewcommand\ulgray[1]{#1}

% To hide todonotes, use: \usepackage[disable]{todonotes} in preamble.tex
% To hide changes, use: \usepackage[final]{changes} in preamble.tex
 % For hiding TODOS, highlights, underlines and callouts

% Add other dependencies in preamble.tex (for cleaner visuals)

\author{Upal Bhattacharya}
\date{}
\title{Ontology LLM Systematic Literature Review}
\begin{document}

\maketitle

\begingroup
    \hypersetup{linkcolor=black}
    \tableofcontents
    \listoftodos
    \pagebreak
\endgroup

\linenumbers

\section{Curation of the Initial List}
\label{sec:initial-list-curation}

\subsection{Keyword Selection}
\label{subsec:keyword-selection}

A conjunction of two groups of keywords: one for \textbf{semantic terms} and
one for \textbf{LLM terms} is used. Keywords for both groups are obtained by
1) selecting/extracting/inferring (from the title, abstract of Introduction
section) of 25 articles that were manually searched for on Google Scholar and
found to be relevant and, 2) from keywords used in the systematic literature
review in \citet{li2025LargeLanguageModels}

A list of the semantic terms found from the 25 manually identified articles is
given below:

\begin{itemize}
  \item \citet{babaei2023Llms4olLargeLanguage}: Ontologies, Ontology Learning
  \item \citet{babaei2025Llms4omMatchingOntologies}: Ontology Matching,
    Ontology Alignment
  \item \citet{li2025LargeLanguageModels}: Ontology Engineering, Ontology
    Development, \textit{ontolog*, ontology development,
    vocabulary}
  \item \citet{mai2024DoLlmsReally}: ontology learning
  \item \citet{bakker2024OntologyLearningText}: ontology learning
  \item \citet{chen2023ContextualSemanticEmbeddings}: ontology embedding,
    subsumption prediction, ontology alignment, OWL
  \item \citet{chen2023PromptingOrFine}: taxonomy construction
  \item \citet{dong2024LanguageModelBased}: ontology enrichment, concept
    placement
  \item \citet{doumanas2024IntegratingLlmsIn}: Ontology Engineering
  \item \citet{du2024ShortReviewOntology}: (\textit{inferred from title})
    Ontology Learning
  \item \citet{funk2023TowardsOntologyConstruction}: (\textit{inferred from
      title, introduction}) ontology construction, ontology engineering
  \item \citet{garijo2024LlmsOntologyEngineering}: Ontology Engineering
  \item \citet{he2023ExploringLargeLanguage}: Ontology Alignment, Ontology
    Matching 
  \item \citet{he2023LanguageModelAnalysis}: (\textit{inferred from task
      classification}) Ontology Alignment
  \item \citet{jain2022DistillingHypernymyRelations}: (\textit{inferred from
      task}) Ontology learning, taxonomy learning
  \item \citet{kommineni2024HumanExpertsMachines}: Ontology
  \item \citet{lippolis2025OntologyGenerationUsing}: Ontology, Ontology
    Engineering
  \item \citet{lo2024EndEndOntology}: (\textit{inferred from title}) Ontology
    learning
  \item \citet{mateiu2023OntologyEngineeringWith}: (\textit{inferred from
      title}) Ontology Engineering
  \item \citet{norouzi2024OntologyPopulationUsing}: ontology modeling
  \item \citet{snijder2024AdvancingOntologyAlignment}: (\textit{inferred from
      abstract}) ontology matching, ontology alignment
  \item \citet{fathallah2024NeonGptLarge}: Ontology Modelling
  \item \citet{tsaneva2024LlmDrivenOntology}: Ontology Evaluation
  \item \citet{yang2024LlmSupportedApproach}: Ontology (\textit{inferred from
      title}) ontology construction
  \item \citet{zeginis2024ApplyingOntologyAware}: (\textit{inferred from
      abstract}) ontology construction
\end{itemize}

The LLM terms obtained from the same set of 25 articles is as follows:

\begin{itemize}
  \item \citet{babaei2023Llms4olLargeLanguage}: Large Language Models, LLMs
  \item \citet{babaei2025Llms4omMatchingOntologies}: Large Language Models
  \item \citet{li2025LargeLanguageModels}: Large Language Models,
    \textit{Language Model, LM, LLM*}
  \item \citet{mai2024DoLlmsReally}: LLMs
  \item \citet{bakker2024OntologyLearningText}: large language models
  \item \citet{chen2023ContextualSemanticEmbeddings}: Pre-trained language
    model, BERT
  \item \citet{chen2023PromptingOrFine}: large language models
  \item \citet{dong2024LanguageModelBased}: Pre-trained Languagem Models,
    Large Language Models
  \item \citet{doumanas2024IntegratingLlmsIn}: Large Language Models
  \item \citet{du2024ShortReviewOntology}: (\textit{inferred from title})
    Large Language Models 
  \item \citet{funk2023TowardsOntologyConstruction}: (\textit{inferred from
      abstract}) large language model
  \item \citet{garijo2024LlmsOntologyEngineering}: Large Language Models
  \item \citet{he2023ExploringLargeLanguage}: Large Language Model
  \item \citet{jain2022DistillingHypernymyRelations}: (\textit{inferred from
      text}) language model
  \item \citet{kommineni2024HumanExpertsMachines}: Large Language Models
  \item \citet{lippolis2025OntologyGenerationUsing}: Large Language Models
  \item \citet{lo2024EndEndOntology}: (\textit{inferred from title}) Large
    Language Model
  \item \citet{mateiu2023OntologyEngineeringWith}: (\textit{inferred from
      title}) Large Language Models
  \item \citet{norouzi2024OntologyPopulationUsing}: large language models
  \item \citet{snijder2024AdvancingOntologyAlignment}: (\textit{inferred from
      abstract}) large language model
  \item \citet{fathallah2024NeonGptLarge}: Large Language Models
  \item \citet{tsaneva2024LlmDrivenOntology}: Large Language Models
  \item \citet{yang2024LlmSupportedApproach}: LLM
  \item \citet{zeginis2024ApplyingOntologyAware}: (\textit{inferred from
      abstract}) large language model
\end{itemize}
Based on the observed patterns, the semantic keywords selected are:
\begin{lstlisting}
  ontolog* OR
  ontology development OR
  ontology engineering OR
  ontology learning OR
  ontology alignment OR
  ontology matching
\end{lstlisting}

and the LLM keywords are:
\begin{lstlisting}
  LLM* OR
  large language model* or
  language model*
\end{lstlisting}

\subsection{Sources}
\label{subsec:sources}

All searches were done on 2025-09-11

\subsubsection{IEEEXplore}
\label{subsubsec:ieeexplore}

Hits: 496

\begin{figure}[H]
\begin{lstlisting}
  ("All Metadata":"ontolog*")
   AND
  ("All Metadata":"LLM*" OR
   "All Metadata":"language model*")
\end{lstlisting}
  \label{fig:ieeexplore-query}
  \caption{Search Query for IEEEXplore}
\end{figure}

\subsubsection{ScienceDirect}
\label{subsubsec:sciencedirect}

Hits: 5691 \newline

\begin{figure}[H]
\begin{lstlisting}
  ("ontology")
   AND
  ("LLM" OR "language model")
\end{lstlisting}
  \label{fig:sciencedirect-query}
  \caption{Search Query for ScienceDirect}
\end{figure}

\subsubsection{Scopus}
\label{subsubsec:scopus}

Hits: 1280 \newline

\begin{figure}[H]
\begin{lstlisting}
  TITLE-ABS-KEY ( "ontolog*" ) AND
  TITLE-ABS-KEY ( "LLM*" OR "language model*" ) AND
  PUBYEAR > 2017 AND PUBYEAR < 2026
\end{lstlisting}
  \label{fig:scopus-query}
  \caption{Search Query for Scopus}
\end{figure}

\subsubsection{ACL Anthology}
\label{subsubsec:acl}

Hits: 395 \newline

Querying done using the Python API

\subsubsection{ACM Digital Library}
\label{subsubsec:acm}

Hits: 1965

Although the search query says 7299 hits, it stops providing any results after
2000 papers with the error message in Figure
\ref{fig:acm-search-limit-error-message} 

\begin{figure}[H]
\begin{lstlisting}
In order to show you the most relevant results, we have omitted some entries
very similar to the 2000 already displayed.

Please refine the results either via the facet filters on the left or edit
your search in Advanced Search via the button at the top.
\end{lstlisting}
  \caption{Error Message from ACM Digital Library}
  \label{fig:acm-search-limit-error-message}
\end{figure}

\begin{figure}[H]
\begin{lstlisting}
[All: ontolog*]
AND
[
  [All: llm*] OR
  [All: language model*]
]
AND
[E-Publication Date: (01/01/2018 TO 12/31/2025)]
\end{lstlisting}
  \label{fig:acm-query}
  \caption{Search Query for ACM}
\end{figure}

The searched was done ``Anywhere''.

% \subsubsection{ArXiV}
% \label{subsubsec:arxiv}

% Hits: 857 (from website); 900(647) (from API)

% The arXiv API is the specified method of extracting information from
% arXiv. Using the Python Wrapper for the arXiv API
% (\href{https://github.com/lukasschwab/arxiv.py}{GitHub:
%   lukasschwab/arxiv.py}), the IDs of the first 900 hits were
% extracted. Unusual behaviour of the API or the wrapper led to inconsistent
% results across repeated queries. This was accounted for by replicating the
% parameters on the website in the API calls (sort order and number of
% responses). The API was utilized to return IDs until it did not find anymore
% for the provided query.

\subsection{Deduplication and Removal of irrelevant articles}
\label{subsec:deduplication}

\href{https://rayyan.ai}{Rayyan} and \href{https://www.zotero.org}{Zotero}
were used to process the initial list of articles from all data
sources. Deduplication was performed initially using the
\href{https://chenglongma.com/zoplicate/}{zoplicate} plugin of Zotero and
thereafter using Rayyan. Conference proceedings, books, book chapters and some
reviews were removed using keyword-based matching.

\subsection{Initial Screening}
\label{subsec:initial-screening}

\subsubsection{Exclusion Criteria}
\label{subsubsec:exclusion-criteria}

\begin{itemize}
  \item Is not published between 2018 and 2025
  \item The language of the article is not English
  \item Is a Book/Book Chapter/Survey/Review/Systematic Literature Review
  \item Is not a peer-reviewed article
  \item Is not a quantitative study
  \item Is not an application of LLMs for ontology engineering (not relevant)
\end{itemize}

\subsubsection{Inclusion Criteria}
\label{subsubsec:inclusion-criteria}

\begin{itemize}
  \item Is a quantitative experiment about the application of LLMs for
    ontology engineering, from any phase of the ontology engineering lifecycle
  \end{itemize}

\subsubsection{Explanations for some exclusions}
\label{subsubsec:expln-exclusion}

\begin{itemize}
  \item \textbf{Gene Ontology (GO) Enrichment, Human Phenotype Ontology (HPO)
    Annotation, Event Extraction, Information Extraction (IE) , Dialogue State
    Tracking (DST), other annotation tasks:} All articles dealing with such
    tasks work closely with ontologies to classify new information with an
    emphasis on hierarchy. These can be viewed as ontology population to some
    extent but:
    \begin{itemize}
      \item Life Sciences and Medical Ontologies do not contain individuals by
        design making such population not correct
      \item The classification is used to tag well-known structured
        information in clinical notes, user queries, etc. for better
        prediction e.g. recommendation systems. Such tagging of entity types
        is not performed with the intention of 'enriching' the underlying
        ontology with this newly found information for future ease-of-use.
    \end{itemize}
  \item \textbf{Entity Linking:} Entity linking, once again is more
    reminiscent of ontology population but does not actually enrich/update the
    underlying ontology to whose concepts new terms are linked.
  \end{itemize}
  
\subsubsection{Undecided topics}
\label{subsubsec:endecided}
\begin{itemize}
  \item Dialogue ontology construction: How does this differ from dialogue
    state tracking?
  \end{itemize}  
% \subsection{Seed List of Articles}
% \label{subsec:seed-list}

% To validate whether the provided keywords and sources capture relevant
% articles, a seed list of articles is curated. The initial list of articles
% obtained after curation from all sources and removal of duplicates is tested
% for the presence of articles from the seed list. If $90\%$ of the articles
% from the seed list are present in the curated initial list, the list is
% considered valid. Otherwise, we re-evaluate the set of keywords, with special
% emphasis on keywords from articles in the seed list not captured in the
% initial list.

% The seed list consists of the 25 manually identified relevant articles as
% mentioned in Section \ref{subsec:keyword-selection} and of all relevant
% articles from the list of references in the systematic literature review in
% \citet{li2025LargeLanguageModels}. Relevance is determined by considering
% articles published after 2017 and retaining only those articles that contain
% the selected keywords in their title, abstract or Introduction sections (done
% manually). Combining the relevant list of citations from
% \citet{li2025LargeLanguageModels} along with the 25 manually identified
% documents creates a seed list of 60 articles (after removal of duplicates).
\section{Data Extraction and Analysis Variables}
\label{sec:data-extraction-analysis-variables}

\subsection{Data Extraction}
\label{subsec:data-extraction}

Data to catalog for each selected paper:
\begin{enumerate}
  \item \textbf{Domain}: Life Sciences, Medical Sciences, Robotics, Building
    Management, Data Management, etc.
  \item \textbf{Ontologies used}
  \item \textbf{Ontology Engineering Task}
  \item \textbf{LLMs used}
  \item \textbf{LLM method used}: zero/few-shot prompting, RAG, fine-tuning, etc.
  \item \textbf{Task(?)}: Classification, Retrieval, Generation, etc.
  \item \textbf{NLP Metrics}
  \item \textbf{Ontology Metrics}
\end{enumerate}

\subsection{Analysis/Questions over extracted Data}
\label{subsec:analysis-over-data}
\begin{enumerate}
  \item \textbf{Domain-based change in strategy over time}: Medical sciences
    uses a lot of RAG-based approaches now but in the past it focused on
    fine-tuning and transfer learning, etc.
  \item 
\end{enumerate}
  
  

\subsection{Relevant Variables}
\label{subsec:relevant-variables}

\begin{itemize}
\item OL Layer-Cake (Default Task categorization) \textit{add citation here}
\item NLP approach-based categorization of OL tasks.
  From \cite{du2024ShortReviewOntology}
\item Prompting vs. Fine-tuning methods
\item Low-resource approaches
\item Ontology learning tasks vs. ontology enrichment tasks and unifying the
  different task classifications
\item LLMs for knowledge graph/base construction/population/updation without
  refernce to underlying ontologies
\end{itemize}

\subsection{Other Interesting Variables}
\label{subsec:other-variables}

\begin{itemize}
  \item LLMs replacing rule-based (ontology) systems (dialogue state tracking)
    vs. LLMs supporting rule-based systems (clinical note annotation) vs. LLMs
    enriching rule-based systems (enrichment)
  \end{itemize}



\bibliographystyle{splncs04nat}
\bibliography{bibliography}
\end{document}

